\begin{multicols}{2}[\chapter{Introduction}]

\section{Idea}

The idea to create a new server configuration tool was born out of a simple concideration,
the configuration of each new server is always the same for the same stack, and the stack
is always similar, i.e. web server, database server, email server, DNS server.
So, I asked myself, what if I could just tell the computer to install and configure me the stack
on the server, without for me to actually have any knowledge how to configure it.
The user should be able to tell the computer to just install and setup, for example, a web-server
without to have the actual knowledge how to do it. All the knowledge how to 
install and configure the server should be contained in the application that is 
installing and configures the server.

\section{Expert Knowledge}

\AppName{} will contain expert knowledge to install and configure the server
with the needed services and applications. The expert knowledge will contain 
best practices in regards to security and it will updated regularly to
fix bugs and to react to new security issues.
To ensure a constant flow of updated expert knowledge, the OSGi technology\footnote{\url{https://www.osgi.org}}
will be used for \AppName{}. OSGi container allows for seamlessly updates of
the modules that contain the expert knowledge from a central repository.

\section{OSGi Container}

There are multiple OSGi container implementations\footnote{Apache Felix, Equinox OSGi, Knopflerfish, etc.} 
and \AppName{} will be able
to run on those, but also it will provide its own container, Karaf\footnote{\url{http://karaf.apache.org}}.
Based on the Karaf container, \AppName{} can be installed on a GNU/Linux server
just as any other application and run as a service in the background.
The communication between \AppName{} and the user is done via a SSH connection and the
OSGi console.

\section{Domain Specific Language}

The user is expected to write simple scripts in a domain specific language (DSL)
based on Groovy that communicate the wishes of the user to \AppName{}. A syntax
must be followed that groups the server services into categories, like web server,
DNS server, mail server, etc. and also have application categories like Wordpress,
Drupal, Postfix, MySQL, etc. The user is assumed to have only very basic knowledge 
of those services and applications, for example, the user should know that the MySQL database
server have an administration user, provides databases and have users that have
permissions to create tables in those databases. But this kind of knowledge
is not expert knowledge and is expected from the user to have. Expert knowledge in this
case would be how to install the MySQL server, configure the server, create
databases and users on the server and grant the users permissions to those databases.

\end{multicols}

\begin{figure}[h]
\begin{center}
\includegraphics[width=0.7\textwidth]{sscontrol-overview_en} \\
\captionof{figure}{\AppName{} overview.}
\end{center}
\end{figure}

