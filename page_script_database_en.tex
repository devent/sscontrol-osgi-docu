\section{Database Script}

The database script is used to configure the database server, to specify
the available databases on the server and the users that have access to those
databases.

\subsection[Database Statements]{Statements}

The database statements must start with \qcode{database.with \{ ... \}} and
contains the database statements inside the curly braces.
The statements take either a list of arguments or a map that contains the arguments
as \code{name:=value} pairs.

\subsubsection{debug}

\TheStatement[database:debug]{debug}
\TheStatement*[database!debug]{debug \Arg{name}, [level: \Arg{level}] [file: \Arg{file}] [\dots]}

Sets the debug logging \Arg{level} and logging \Arg{file} for the specified 
logger \Arg{name}. The \Arg{name} identifies the logger of the server and 
the \Arg{level} can be any positive number, zero is normally interpreted as no logging.

\begin{lstlisting}[style=Groovy]
database.with {
    debug "error", level: 1
}
\end{lstlisting}

\begin{lstlisting}[style=Groovy]
debugLogs = []
debugLogs << [name: "error", level: 1]
debugLogs << [name: "slow-queries", level: 1]
debugLogs << [name: "general", level: 1, file: "/var/log/mysql/mysql.log"]
debugLogs.each { database.debug it }
\end{lstlisting}

\subsubsection{bind}

\TheStatement[database:bind]{bind}
\TheStatement*[database!bind]{bind local|all|\Arg{address} [, port: \Arg{port}]}

Sets the binding \Arg{address} and the binding \Arg{port}. Set to \qcode{local}
for the local host address \code{127.0.0.1} or to \qcode{all} to listen to all
hosts.

\begin{lstlisting}[style=Groovy]
database.with {
    bind "192.168.0.1", port: 3306
}
\end{lstlisting}

\subsubsection{admin}

\TheStatement[database:admin]{admin}
\TheStatement*[database!admin]{admin \Arg{name}, password: \Arg{password}}

Sets the administrator \Arg{name} and \Arg{password} for the database server. If no password for
the administrator account was yet set, this administrator name and password is set. The administrator 
user is used to to create users, create databases and set permissions.

\begin{lstlisting}[style=Groovy]
database.with {
    admin 'root', password: 'mysqladminpassword'
}
\end{lstlisting}

\subsubsection{db}

\TheStatement[database:db]{db}
\TheStatement*[database!db]{db \Arg{name} [, charset: \Arg{name}] [, collate: \Arg{name}] [, \{ script \}]}

Creates a new database with the specified database \Arg{name}, 
and, optionally, the character set and collate. If the database was already on 
the server, the character set and collate should be updated to the specified 
character set and collate. The database server needs to support 
the specified character set and collate.

\begin{lstlisting}[style=Groovy]
database.with {
    db "postfixdb", charset: "latin1", collate: "latin1_swedish_ci"
}
\end{lstlisting}

\subsubsection{script}

\TheStatement[database:script]{script}
\TheStatement*[database!script]{script importing: \Arg{resource}}

Imports the script \Arg{resource}. The resource can be a file, URL or URI. 
A string will be interpreted according to the format, if no scheme is used 
the string is assumed to be a local file.

\begin{lstlisting}[style=Groovy]
database.with {
    db "postfixdb" with {
        script importing: "postfixtables.sql"
    }
}
\end{lstlisting}

\subsection[Database Example]{Example}

It is recommended to put the script variables in a standalone file, for example
in ``Globals.groovy'' and load the script variables using 
the ``evaluate'' statement.

\vspace{1em}
\begin{lstlisting}[style=Groovy,caption={Predefined database variables.}]
/**
* Globals.groovy
*/
def databasemap = [
    server: '127.0.0.1',
    port: 3306,
    password: 'somepass',
    phpmyadmin: [
        user: 'phpmyadmin',
        password: 'somepass',
        db: 'phpmyadmindb',
    ],
]
\end{lstlisting}

\begin{lstlisting}[style=Groovy,caption={Database script example.}]
/**
* Database.groovy
*/
evaluate "./Globals.groovy" as File

database.with {
    bind databasemap.server, port: databasemap.port
    admin user: 'root', password: databasemap.password
    db name: databasemap.phpmyadmin.db
    user databasemap.phpmyadmin.user, password: databasemap.phpmyadmin.password with {
        access database: databasemap.phpmyadmin.db
    }
}
\end{lstlisting}
